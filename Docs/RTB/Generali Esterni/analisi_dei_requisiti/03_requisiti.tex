\newcounter{RF1}
\newcounter{RF2}
\newcounter{RF3}
\newcounter{RV1}
\newcounter{RV2}
\newcounter{RV3}
\newcounter{RP1}
\newcounter{RP2}
\newcounter{RP3}
\newcounter{RQ1}
\newcounter{RQ2}
\newcounter{RQ3}

\newcounter{RF}
\newcounter{RV}
\newcounter{RP}
\newcounter{RQ}

\newcounter{M}
\newcounter{D}
\newcounter{O}

\newcommand{\RFM}{
    \stepcounter{M}
    \stepcounter{RF}
    \stepcounter{RF1}
    RF\arabic{RF}
}
\newcommand{\RFD}{
    \stepcounter{D}
    \stepcounter{RF}
    \stepcounter{RF2}
    RF\arabic{RF}
}
\newcommand{\RFO}{
    \stepcounter{O}
    \stepcounter{RF}
    \stepcounter{RF3}
    RF\arabic{RF}
}
\newcommand{\RVM}{
    \stepcounter{M}
    \stepcounter{RV}
    \stepcounter{RV1}
    RV\arabic{RV}
}
\newcommand{\RVD}{
    \stepcounter{D}
    \stepcounter{RV}
    \stepcounter{RV2}
    RV\arabic{RV}
}
\newcommand{\RVO}{
    \stepcounter{O}
    \stepcounter{RV}
    \stepcounter{RV3}
    RV\arabic{RV}
}
\newcommand{\RPM}{
    \stepcounter{M}
    \stepcounter{RP}
    \stepcounter{RP1}
    RP\arabic{RP}
}
\newcommand{\RPD}{
    \stepcounter{D}
    \stepcounter{RP}
    \stepcounter{RP2}
    RP\arabic{RP}
}
\newcommand{\RPO}{
    \stepcounter{O}
    \stepcounter{RP}
    \stepcounter{RP3}
    RP\arabic{RP}
}
\newcommand{\RQM}{
    \stepcounter{M}
    \stepcounter{RQ}
    \stepcounter{RQ1}
    RQ\arabic{RQ}
}
\newcommand{\RQD}{
    \stepcounter{D}
    \stepcounter{RQ}
    \stepcounter{RQ2}
    RQ\arabic{RQ}
}
\newcommand{\RQO}{
    \stepcounter{O}
    \stepcounter{RQ}
    \stepcounter{RQ3}
    RQ\arabic{RQ}
}


\newcommand{\requirementsTable}[1]{

\renewcommand{\arraystretch}{1.5}
\rowcolors{2}{pari}{dispari}
\begin{longtable}{ %0.87
		>{\centering}M{0.15\textwidth} 
		>{\centering}M{0.5\textwidth}
		>{\centering}M{0.20\textwidth}
		>{\centering\arraybackslash}M{0.15\textwidth} 
		 }
	\rowcolorhead
	\headertitle{Requisito} &
	\centering \headertitle{Descrizione} &	
	\headertitle{Classificazione} &
	\headertitle{Fonte}
	\endfirsthead	
	\endhead
	
	#1

\end{longtable}
\vspace{2em}

}

\newcommand{\SrcToReqTable}[1]{

\renewcommand{\arraystretch}{1.5}
\rowcolors{2}{pari}{dispari}
\begin{longtable}{ %0.87
		>{\centering}M{0.5\textwidth} 
		>{\centering\arraybackslash}M{0.20\textwidth} 
		 }
	\rowcolorhead
	\headertitle{Fonte} &
	\centering\headertitle{Requisito}
	\endfirsthead	
	\endhead
	
	#1

\end{longtable}
\vspace{2em}

}

\newcommand{\ReqToUCTable}[1]{

\renewcommand{\arraystretch}{1.5}
\rowcolors{2}{pari}{dispari}
\begin{longtable}{ %0.87
		>{\centering}M{0.5\textwidth} 
		>{\centering\arraybackslash}M{0.20\textwidth} 
		 }
	\rowcolorhead
	\headertitle{Requisito} &
	\centering\headertitle{Fonte}
	\endfirsthead	
	\endhead
	
	#1

\end{longtable}
\vspace{2em}

}

\newcommand{\requirementsSummaryTable}[1]{

\renewcommand{\arraystretch}{1.5}
\rowcolors{2}{pari}{dispari}
\begin{longtable}{ %0.87
		>{\centering}M{0.3\textwidth} 
		>{\centering}M{0.15\textwidth}
		>{\centering}M{0.15\textwidth}
		>{\centering}M{0.15\textwidth}
		>{\centering\arraybackslash}M{0.15\textwidth} 
		 }
	\rowcolorhead
	\headertitle{Tipologia} &
	\centering \headertitle{Obbligatorio} &	
	\headertitle{Desiderabile} &
	\headertitle{Opzionale} &
	\headertitle{Totale}
	\endfirsthead	
	\endhead
	
	#1

\end{longtable}
\vspace{2em}

}

\section{Requisiti}
\subsection{Introduzione}
Sono stati definiti dei requisiti codificati in base all’ambito di competenza e ad un numero seriale per
tenerne meglio traccia. Nelle tabelle sottostanti sono forniti descrizione e classificazione di ciascun
requisito. 
Per la codifica dei requisiti si faccia riferimento alle \textit{Norme di progetto}.
\subsection{Requisiti funzionali}
\requirementsTable{
    % VISUALIZZAZIONE DATASET
    \RFM & L’utente deve poter visualizzare i dataset  & Obbligatorio & UC1 \tabularnewline
    \RFM & L’utente deve poter visualizzare le informazioni primarie dei dataset  & Obbligatorio & UC1.1\par UC1.1.1\par UC1.1.2 \tabularnewline
    \RFM & L’utente deve poter visualizzare i dettagli dei dataset  & Obbligatorio  & UC2\par UC2.1 \tabularnewline

    % CARICAMENTO DATASET
    \RFM & L’utente deve poter caricare il dataset &  Obbligatorio &UC3\tabularnewline
    \RFM & L’utente deve ricevere un errore se il caricamento di un dataset fallisce  & Obbligatorio& UC4\tabularnewline

    % VISUALIZZAZIONE DEI DATI IN TABELLA
    \RFM & L’utente deve poter visualizzare i dati del dataset selezionato in forma tabellare & Obbligatorio   & UC5\tabularnewline
    \RFM & L’utente deve poter visualizzare le intestazioni della tabella  & Obbligatorio  & UC5.1 \tabularnewline
    \RFM & L’utente deve poter visualizzare il valore del dato in una cella della tabella  & Obbligatorio  & UC5.2 \tabularnewline
    
    % VISUALIZZAZIONE DEI DATI SUL GRAFICO
    \RFM & L’utente deve poter visualizzare i dati del dataset selezionato in forma di grafico tridimensionale a barre verticali  & Obbligatorio  &  UC6\tabularnewline
    \RFM & L’utente deve poter visualizzare gli assi del grafico tridimensionale  & Obbligatorio   &  UC6.1\tabularnewline
    \RFM & L’utente deve poter visualizzare l'asse X del grafico tridimensionale   & Obbligatorio  & UC6.1.1 \tabularnewline
    \RFM & L’utente deve poter visualizzare l'asse Y del grafico tridimensionale   & Obbligatorio  & UC6.1.2 \tabularnewline
    \RFM & L’utente deve poter visualizzare l'asse Z del grafico tridimensionale   & Obbligatorio  & UC6.1.3 \tabularnewline

    
    % TELECAMERA
    \RFM & L’utente deve poter muovere la telecamera orizzontalmente & Obbligatorio  &  UC7\tabularnewline
    \RFM & L’utente deve poter muovere la telecamera verticalmente  & Obbligatorio &  UC8\tabularnewline
    
    % TELECAMERA
    \RFM & L’utente deve poter ruotare il grafico attorno all'asse X   & Obbligatorio & UC9\tabularnewline
    \RFM & L’utente deve poter ruotare il grafico attorno all'asse Y  & Obbligatorio  & UC10\tabularnewline
    \RFM & L’utente deve poter ruotare il grafico attorno all'asse Z   & Obbligatorio & UC11\tabularnewline
    
    %ZOOM
    \RFM & L’utente deve poter eseguire un’azione di zoom indipendentemente dalla posizione della telecamera & Obbligatorio & UC12\tabularnewline

    %RIPOSIONNAMENTO
    \RFM & L’utente deve poter riposizionare la telecamera alla sua posizione iniziale  & Obbligatorio  & UC13\tabularnewline

    %VISUALIZZAZIONE PRIMO PIANO
    \RFM & L’utente deve poter visualizzare una singola barra del grafico in primo piano  & Obbligatorio & UC14\tabularnewline

    %VISUALIZZAZIONE DETTAGLI
    \RFM & L’utente deve poter visualizzare i dettagli di una barra del grafico  & Obbligatorio & UC15\par UC15.1\par UC15.2\tabularnewline

    %FILTRI
    \RFM & L’utente deve poter filtrare un numero N dei valori più alti & Obbligatorio & UC16\par UC16.1\tabularnewline
    \RFM & L’utente deve poter filtrare un numero N dei valori più bassi & Obbligatorio & UC16\par UC16.2\tabularnewline
    \RFM & L’utente deve poter inserire un numero per effettuare un filtraggio & Obbligatorio & UC16\par UC16.3\tabularnewline
    \RFM & L’utente deve poter filtrare i valori inferiori o uguali rispetto a un valore del grafico & Obbligatorio & UC17\par UC17.1\par UC17.1.1\tabularnewline
    \RFM & L’utente deve poter filtrare i valori inferiori o uguali rispetto al valor medio globale & Obbligatorio & UC17\par UC17.1\par UC17.1.2\tabularnewline
    \RFM & L’utente deve poter filtrare i valori superiori o uguali rispetto a un valore del grafico & Obbligatorio & UC17\par UC17.2\par UC17.2.1\tabularnewline
    \RFM & L’utente deve poter filtrare i valori superiori o uguali rispetto al valor medio globale & Obbligatorio & UC17\par UC17.2\par UC17.2.2\tabularnewline

    %RESET
    \RFM & L’utente deve poter ripristinare lo stato del grafico
    rispetto ai precedenti filtri applicati   & Obbligatorio & UC18\tabularnewline

    %VALOR MEDIO
    \RFM & L’utente deve poter visualizzare un piano parallelo alla base che rappresenti il valor medio globale & Obbligatorio & UC19\par UC19.1\tabularnewline
    \RFD & L’utente deve poter visualizzare un piano parallelo alla base che rappresenti la media dei valori relativi a un etichetta & Desiderabile & UC19\par UC19.2\tabularnewline

    %COLORE
    \RFO & Le barre del grafico 3D devono avere una colorazione appropriata che renda i relativi valori facilmente e velocemente comprensibili  & Opzionale & VE 2024-12-19 \tabularnewline

    }

   
\subsection{Requisiti qualitativi}
    
\requirementsTable{
    \RQM & Il software deve essere sviluppato seguendo le metriche e il modello di qualità descritti nel documento \textit{Norme di progetto}  & Obbligatorio  & Decisione interna \tabularnewline
    \RQM & Il software deve essere sviluppato pubblicando il codice sorgente nel repository Github dedicato  & Obbligatorio  & Decisione interna \tabularnewline
    \RQM & Il software deve essere sviluppato in modo tale da supportare grandi volumi di dati   & Obbligatorio & Capitolato \tabularnewline
    \RQM & L'architettura del software deve permettere, con agilità, di aggiungere nuove funzionalità e possibilità di iterazione con il grafico   & Obbligatorio  & Capitolato \tabularnewline
    \RQM & La suite di test deve essere robusta per garantire quanto specificato nel RQ4  & Obbligatorio  & Capitolato \tabularnewline
    \RQD & Il software deve essere testato attraverso test di tipo e2e  & Desiderabile  & Capitolato \tabularnewline  
}
 \pagebreak
\subsection{Requisiti di vincolo}
\requirementsTable{
    \RVM & Il software deve essere sviluppato utilizzando Typescript come linguaggio di programmazione primario  & Obbligatorio  & Decisione interna \tabularnewline
    \RVM & Il software deve essere sviluppato utilizzando la libreria React per la creazione di un'interfaccia visiva semplice  & Obbligatorio  & Decisione interna \tabularnewline
    \RVM & Il software deve essere sviluppato utilizzando la libreria Three.js   & Obbligatorio & Decisione interna \tabularnewline
    \RVM & Il software deve essere compatibile dalla versione 131 del browser Chrome  & Obbligatorio & Decisione interna \tabularnewline
    \RVM & Il software deve essere compatibile dalla versione 133 del browser Firefox  & Obbligatorio & Decisione interna \tabularnewline
    \RVM & Il software deve essere compatibile dalla versione 115 del browser Opera  & Obbligatorio & Decisione interna \tabularnewline
    \RVM & Il software deve essere compatibile dalla versione 131 del browser Microsoft Edge  & Obbligatorio & Decisione interna \tabularnewline
    }
\newpage

\subsection{Requisiti prestazionali}
\requirementsTable{
    \RPM & Il software quando carica un dataset di dimensioni minori o uguali a 100000 non deve avere un tempo di caricamento superiore a 2 secondi  & Obbligatorio & Decisione interna \tabularnewline
    \RPM & Il software quando carica un dataset di dimensioni minori o uguali a 500000 non deve avere un tempo di caricamento superiore a 5 secondi  & Obbligatorio & Decisione interna \tabularnewline
    \RPM & Il software quando carica un dataset di dimensioni minori o uguali a 1000000 non deve avere un tempo di caricamento superiore a 8 secondi  & Obbligatorio & Decisione interna \tabularnewline
    \RPD & Il software quando carica un dataset di dimensioni maggiori a 1000000 non deve avere un tempo di caricamento superiore a 15 secondi  & Desiderabile & Decisione interna \tabularnewline
    \RPM & Il software non deve avere un tempo di risposta superiore a 5 secondi & Obbligatorio & Decisione interna \tabularnewline
}


\subsection{Tracciamento requisiti}

\subsubsection{Fonte - Requisiti}
\SrcToReqTable{
    Capitolato & RQ3\par RQ4\par RQ5\par RQ6\par RV8 \tabularnewline
    Decisione interna & RQ1\par
                        RQ2\par 
                        RV1\par 
                        RV2\par 
                        RV3\par 
                        RV4\par
                        RV5\par
                        RV6\par
                        RV7\par
                        RP1\par
                        RP2\par
                        RP3\par
                        RP4\par
                        RP5\par
                        \tabularnewline
    UC1    & RF1 \tabularnewline
    UC1.1    & RF2 \tabularnewline
    UC1.1.1    & RF2 \tabularnewline
    UC1.1.2    & RF2 \tabularnewline
    UC2    & RF3 \tabularnewline
    UC2.1    & RF3 \tabularnewline
    UC3    & RF4 \tabularnewline
    UC4   & RF5 \tabularnewline
    UC5    & RF6 \tabularnewline
    UC5.1    & RF7 \tabularnewline
    UC5.2    & RF8 \tabularnewline
    UC6    & RF9 \tabularnewline
    UC6.1    & RF10 \tabularnewline
    UC6.1.1    & RF11 \tabularnewline
    UC6.1.2    & RF12 \tabularnewline
    UC6.1.3    & RF13 \tabularnewline
    UC7    & RF14 \tabularnewline
    UC8   & RF15 \tabularnewline
    UC9    & RF16 \tabularnewline
    UC10    & RF17 \tabularnewline
    UC11   & RF18 \tabularnewline
    UC12    & RF19 \tabularnewline
    UC13    & RF20 \tabularnewline
    UC14   & RF21 \tabularnewline
    UC15    & RF22 \tabularnewline
    UC15.1    & R22 \tabularnewline
    UC15.2    & RF22 \tabularnewline
    UC16    & RF23\par RF24\par RF25 \tabularnewline
    UC16.1    & RF23 \tabularnewline
    UC16.2    & RF24 \tabularnewline
    UC16.3    & RF25 \tabularnewline
    UC17    & RF26\par RF27\par RF28\par RF29 \tabularnewline
    UC17.1    & RF26\par RF27 \tabularnewline
    UC17.1.1    & RF26 \tabularnewline
    UC17.1.2    & RF27 \tabularnewline
    UC17.2    & RF28\par RF29 \tabularnewline
    UC17.2.1    & RF28 \tabularnewline
    UC17.2.2    & RF29 \tabularnewline
    UC18   & RF30 \tabularnewline
    UC19   & RF31\par RF32 \tabularnewline
    UC19.1    & RF31 \tabularnewline
    UC19.2    & RF32 \tabularnewline
    VE 2024-12-19  & RF33 \tabularnewline

    }

\subsubsection{Requisiti - Fonti}
\ReqToUCTable{

    RF1 & UC1 \tabularnewline
    RF2 & UC1.1\par UC1.1.1\par UC1.1.2\tabularnewline
    RF3 & UC2\par UC2.1 \tabularnewline
    RF4 & UC3 \tabularnewline
    RF5 & UC4 \tabularnewline
    RF6 & UC5 \tabularnewline
    RF7 & UC5.1 \tabularnewline
    RF8 & UC5.2 \tabularnewline
    RF9 & UC6 \tabularnewline
    RF10 & UC6.1 \tabularnewline
    RF11 & UC6.1.1 \tabularnewline
    RF12 & UC6.1.2 \tabularnewline
    RF13 & UC6.1.3 \tabularnewline
    RF14 & UC7 \tabularnewline
    RF15 & UC8 \tabularnewline
    RF16 & UC9 \tabularnewline
    RF17 & UC10 \tabularnewline
    RF18 & UC11\tabularnewline
    RF19 & UC12 \tabularnewline
    RF20 & UC13 \tabularnewline
    RF21 & UC14 \tabularnewline
    RF22 & UC15\par UC15.1\par UC15.2 \tabularnewline
    RF23 & UC16\par UC16.1 \tabularnewline
    RF24 & UC16\par UC16.2 \tabularnewline
    RF25 & UC16\par UC16.3 \tabularnewline
    RF26 & UC17\par UC17.1\par UC17.1.1 \tabularnewline
    RF27 & UC17\par UC17.1\par UC17.1.2 \tabularnewline
    RF28 & UC17\par UC17.2\par UC17.2.1 \tabularnewline
    RF29 & UC17\par UC17.2\par UC17.2.2 \tabularnewline
    RF30 & UC18 \tabularnewline
    RF31 & UC19\par UC19.1 \tabularnewline
    RF32 & UC19\par UC19.2 \tabularnewline
    RF33 & VE 2024-12-19  \tabularnewline

    RQ1 & Decisione interna\tabularnewline
    RQ2 & Decisione interna\tabularnewline
    RQ3 & Capitolato\tabularnewline
    RQ4 & Capitolato\tabularnewline
    RQ5 & Capitolato\tabularnewline
    RQ6 & Capitolato\tabularnewline

    RV1 & Decisione interna\tabularnewline
    RV2 & Decisione interna\tabularnewline
    RV3 & Decisione interna\tabularnewline
    RV4 & Decisione interna\tabularnewline
    RV5 & Decisione interna\tabularnewline
    RV6 & Decisione interna\tabularnewline
    RV7 & Decisione interna\tabularnewline
    RV7 & Capitolato\tabularnewline

    RP1 & Decisione interna\tabularnewline
    RP2 & Decisione interna\tabularnewline
    RP3 & Decisione interna\tabularnewline
    RP4 & Decisione interna\tabularnewline
    RP5 & Decisione interna\tabularnewline

}

\newcounter{T}

\addtocounter{T}{\value{M}}
\addtocounter{T}{\value{D}}
\addtocounter{T}{\value{O}}



\subsubsection{Riepilogo requisiti}
\requirementsSummaryTable{
    Funzionale & \arabic{RF1} & \arabic{RF2} & \arabic{RF3} & \arabic{RF} \tabularnewline
    Di qualità & \arabic{RQ1} & \arabic{RQ2} & \arabic{RQ3} & \arabic{RQ} \tabularnewline
    Di vincolo & \arabic{RV1} & \arabic{RV2} & \arabic{RV3} & \arabic{RV} \tabularnewline
    Prestazionali & \arabic{RP1} & \arabic{RP2} & \arabic{RP3} & \arabic{RP} \tabularnewline
    \bottomrule
    \textbf{Totale requisiti} & \arabic{M} & \arabic{D} & \arabic{O} & \arabic{T}
    
}

