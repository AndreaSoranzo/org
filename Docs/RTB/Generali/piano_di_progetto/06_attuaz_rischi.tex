\newcommand{\riskMitigation}[1]{%
    \renewcommand{\arraystretch}{1.5}% Modifica l'altezza delle righe
    \rowcolors{2}{pari}{dispari}% Alterna i colori delle righe
    \begin{longtable}{%
        >{\centering\arraybackslash}m{0.22\textwidth}%
        >{\centering\arraybackslash}m{0.38\textwidth}%
        >{\centering\arraybackslash}m{0.38\textwidth}%
    }%
        \rowcolorhead
        \headertitle{Nome} &
        \headertitle{Descrizione} &
        \headertitle{Mitigazione} \\
        \endfirsthead
        \endhead

        #1 % Contenuto dinamico

    \end{longtable}
    \vspace{1em}
}

\section{Attualizzazione dei rischi}
In questa sezione sono riportati i rischi che si sono verificati nel corso del progetto e le relative misure di mitigazione attuate dal gruppo.

\riskMitigation{
    \textbf{Comprensione dei requisiti} & Basandosi sul capitolato, non tutti i requisiti erano chiari. & Si sono svolte delle riunioni con il proponente, per discutere e chiarire dubbi. \\
    \textbf{Inesperienza Tecnologica} & Nella fase iniziale del progetto, e durante lo sviluppo del PoC sono sorti dubbi e difficoltà rispetto all'utilizzo di GitHub e delle tecnologie proposte dall'azienda. & Ogni membro del gruppo ha ricavato del tempo personale da dedicare allo studio delle tecnologie da utilizzare, in modo da essere allineati e procedere con maggiore efficienza. \\
    \textbf{Disponibilità dei componenti} & A causa delle festività, la disponibilità dei componenti del gruppo si è ridotta durante quello sprint. & Nella pianificazione si è tenuto conto di questo fattore, inoltre, lo sprint è durato una settimana in più.\\
    \textbf{Discussioni interne} & Durante gli sprint è capitato di dover prendere delle decisioni o di avere dei dubbi, che andassero risolti prima del meeting interno di fine periodo. & È stato usato il canale di comunicazione asincrono Telegram per poter discutere e arrivare a una conclusione.
}
