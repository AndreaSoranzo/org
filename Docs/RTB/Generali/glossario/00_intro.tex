\section*{Introduzione}
\addcontentsline{toc}{section}{Introduzione}
    Il presente documento costituisce una raccolta di definizioni chiare e precise dei termini non
    comuni utilizzati nella documentazione prodotta dal gruppo. \\
    Negli altri documenti, i termini presenti nel \textit{Glossario} sono contrassegnati dalla lettera g a pedice (es. termine\textsubscript{g}).\\
    Per velocizzare la ricerca di un vocabolo, il \textit{Glossario} è suddiviso in sezioni,
    una per ogni lettera, contenente le parole che hanno come iniziale quella lettera. Le sezioni sono 
    naturalmente disposte in ordine alfabetico.\\
    Per evitare inconsistenze di forma, per i termini stranieri è indicato se sono sostantivi (s.) o aggettivi (agg.).
    Nei sostantivi è specificato se sono comuni (c.) o propri (p.), maschili (m.) o femminili (f.).\\
    Il \textit{Glossario} è redatto in maniera incrementale, in base all'utilizzo
    di nuovi termini da definire.
