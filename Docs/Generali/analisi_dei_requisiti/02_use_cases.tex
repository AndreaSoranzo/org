\newcounter{UC}
\newcounter{SUC}
\newcounter{SSUC}

\newcommand{\resetCounter}[1]{%
    \setcounter{#1}{0}
}

\newcommand{\UseCase}[1]{%
    \refstepcounter{UC}
    \resetCounter{SUC}
    \resetCounter{SSUC}
    \subsection{UC\arabic{UC} - #1}\label{UC:\arabic{UC}}
}

\newcommand{\SubUseCase}[1]{%
    \stepcounter{SUC}
    \subsubsection{UC\arabic{UC}.\arabic{SUC} - #1}
}

\newcommand{\SubSubUseCase}[1]{%
    \stepcounter{SSUC}  
    \paragraph{UC\arabic{UC}.\arabic{SUC}.\arabic{SSUC} - #1}
}

% COME UTILIZZARE I CASI D`USO:
% i comandi di use case vanno utilizzati al posto di subsection, subsubsection e paragraph

% \UseCase => rappresenta lo use case principale (UC1,UC2,UC3)
% \SubUseCase => rappresenta il primo sotto livello dello use case principale (UC1.1,UC2.1,UC3.3)
% \SubSubUseCase => rappresenta il secondo sotto livello dello lo use case principale (UC1.1.1,UC2.1.2,UC3.3.1)

\newcommand{\UCdsc}[5]{
    \begin{itemize}
        \item \textbf{Attore primario:}
         #1
        \item \textbf{Descrizione:} 
         #2
        \item \textbf{Precondizioni:}
         #3
        \item \textbf{Postcondizioni:}
        #4
        \item \textbf{Scenario principale:} 
        #5
    \end{itemize}
}


\GetTitleStringSetup{expand}
\section{Casi d'uso}
% \subsection{Attore}
% Poiché per lo svolgimento del progetto non è necessario gestire permessi differenti per l'accesso alle funzionalità, l'attore che interagisce con il nostro software è unico, denominato "Utente".\\
% \textbf{Utente:} soggetto che utilizza la web application, sfruttandone le funzionalità.

\UseCase{Visualizza lista dataset}
    \begin{figure}[h!]
        \centering
        \includegraphics[scale=0.55]{template/images/UC1.png}
        \caption{\nameref*{UC:\arabic{UC}}}
    \end{figure}
    \UCdsc
    { % ATTORE
        \begin{itemize}
            \item Utente in homepage.
        \end{itemize}
    }
    { % DESCRIZIONE
        \begin{itemize}
            \item All'avvio del sistema l'utente visualizza la completa lista dei dataset proposti.
        \end{itemize}
    }
    { % PRECONDIZIONI
        \begin{itemize}
            \item L'utente ha accesso all'applicazione.
        \end{itemize}
    }
    { % POSTCONDIZIONI
        \begin{itemize}
            \item L'utente vede la lista dei dataset disponibili.
        \end{itemize}
    }
    { % SCENARIO PRIMARIO
        \begin{itemize}
            \item L'utente avvia il sistema.
        \end{itemize}
    }


    \newpage

    \SubUseCase{Visualizza singolo dataset}
    \UCdsc
    { % ATTORE
        \begin{itemize}
            \item Utente in homepage.
        \end{itemize}
    }
    { % DESCRIZIONE
        \begin{itemize}
            \item Visualizzazione dei singoli dataset di cui la lista è composta.
        \end{itemize}
    }
    { % PRECONDIZIONI
        \begin{itemize}
            \item La lista dei dataset è visibile.
        \end{itemize}
    }
    { % POSTCONDIZIONI
        \begin{itemize}
            \item Il dataset è visibile nella lista dei dataset.
        \end{itemize}
    }
    { % SCENARIO PRIMARIO
        \begin{itemize}
            \item Nessuna azione richiesta.
        \end{itemize}
    }

    \SubSubUseCase{Visualizza nome dataset}
    \UCdsc
    { % ATTORE
        \begin{itemize}
            \item Utente in homepage.
        \end{itemize}
    }
    { % DESCRIZIONE
        \begin{itemize}
            \item Visualizzazione del nome del dataset. Quest'ultimo è considerato come un identificativo che permette di distinguere i diversi dataset.
        \end{itemize}
    }
    { % PRECONDIZIONI
        \begin{itemize}
            \item Il dataset è presente della lista dei dataset.
        \end{itemize}
    }
    { % POSTCONDIZIONI
        \begin{itemize}
            \item Il nome del dataset è visibile.
        \end{itemize}
    }
    { % SCENARIO PRIMARIO
        \begin{itemize}
            \item Nessuna azione richiesta.
        \end{itemize}
    }


    \SubSubUseCase{Visualizza dimensione dataset}
    \UCdsc
    { % ATTORE
        \begin{itemize}
            \item Utente in homepage.
        \end{itemize}
    }
    { % DESCRIZIONE
        \begin{itemize}
            \item Visualizzazione della dimensione del dataset.
        \end{itemize}
    }
    { % PRECONDIZIONI
        \begin{itemize}
            \item Il dataset è presente della lista dei dataset.
        \end{itemize}
    }
    { % POSTCONDIZIONI
        \begin{itemize}
            \item La dimensione del dataset è visibile.
        \end{itemize}
    }
    { % SCENARIO PRIMARIO
        \begin{itemize}
            \item Nessuna azione richiesta.
        \end{itemize}
    }


\UseCase{Visualizza dettaglio dataset}
    \begin{figure}[h!]
        \centering
        \includegraphics[scale=0.6]{template/images/UC2.png}
        \caption{\nameref*{UC:\arabic{UC}}}
    \end{figure}
    \UCdsc
    { % ATTORE
        \begin{itemize}
            \item Utente in homepage.
        \end{itemize}
    }
    { % DESCRIZIONE
        \begin{itemize}
            \item  L'utente può visualizzare i dettagli di un dataset, ossia le sue informazioni aggiuntive.
        \end{itemize}
    }
    { % PRECONDIZIONI
        \begin{itemize}
            \item Il dataset deve essere un dataset proposto.
        \end{itemize}
    }
    { % POSTCONDIZIONI
        \begin{itemize}
            \item I dettagli del dataset sono visibili.
        \end{itemize}
    }
    { % SCENARIO PRIMARIO
        \begin{itemize}
            \item L'utente interagisce con il sistema espandendo la vista del singolo dataset.
        \end{itemize}
    }
    

    \SubUseCase{Visualizza descrizione dataset}
    \UCdsc
    { % ATTORE
        \begin{itemize}
            \item Utente in homepage.
        \end{itemize}
    }
    { % DESCRIZIONE
        \begin{itemize}
            \item   Visualizzazione della descrizione del dataset.
        \end{itemize}
    }
    { % PRECONDIZIONI
        \begin{itemize}
            \item I dettagli del dataset sono visibili.
        \end{itemize}
    }
    { % POSTCONDIZIONI
        \begin{itemize}
            \item La descrizione del dataset è visibile.
        \end{itemize}
    }
    { % SCENARIO PRIMARIO
        \begin{itemize}
            \item Nessuna azione richiesta.
        \end{itemize}
    }



\UseCase{Carica dataset}
\begin{figure}[h!]
    \centering
    \includegraphics[scale=0.6]{template/images/UC3.png}
    \caption{\nameref*{UC:\arabic{UC}}}
\end{figure}
    \UCdsc
    { % ATTORE
        \begin{itemize}
            \item Utente in homepage.
        \end{itemize}
    }
    { % DESCRIZIONE
        \begin{itemize}
            \item  L'utente decide il dataset che vuole visualizzare andando quindi a caricare tutti i dati e valori nel sistema.
        \end{itemize}
    }
    { % PRECONDIZIONI
        \begin{itemize}
            \item La lista dei dataset disponibili è visibile.
        \end{itemize}
    }
    { % POSTCONDIZIONI
        \begin{itemize}
            \item Il dataset viene caricato nel sistema;
            \item L'utente entra nell'ambiente 3D.
        \end{itemize}
    }
    { % SCENARIO PRIMARIO
        \begin{itemize}
            \item L'utente seleziona un dataset tra quelli proposti nella lista.
        \end{itemize}
        % ESTENSIONE
        \item \textbf{Estensione:} \begin{itemize}
            \item UC4 - Visualizza errore nel caricamento.
        \end{itemize}
    }


\newpage

\UseCase{Visualizza errore nel caricamento}
    \UCdsc
    { % ATTORE
        \begin{itemize}
            \item Utente in homepage.
        \end{itemize}
    }
    { % DESCRIZIONE
        \begin{itemize}
            \item  Visualizzazione di un messaggio di errore "Dataset non disponibile" con una motivazione a seguire.
                    Il caricamento potrebbe fallire perchè il dataset selezionato potrebbe essere
                    non disponibile in quel momento o ci potrebbe essere un errore di connesione.
        \end{itemize}
    }
    { % PRECONDIZIONI
        \begin{itemize}
            \item L'utente ha selezionato un dataset;
            \item Il dataset selezionato non è disponibile.
        \end{itemize}
    }
    { % POSTCONDIZIONI
        \begin{itemize}
            \item Il dataset non viene caricato nel sistema;
            \item L'utente torna alla vista della lista di dataset.
        \end{itemize}
    }
    { % SCENARIO PRIMARIO
        \begin{itemize}
            \item Nessuna azione richiesta.
        \end{itemize}
    }



\UseCase{Visualizza dataset in forma tabellare}
\begin{figure}[h!]
    \centering
    \includegraphics[scale=0.7]{template/images/UC5.png}
    \caption{\nameref*{UC:\arabic{UC}}}
\end{figure}
\UCdsc
    { % ATTORE
        \begin{itemize}
            \item Utente in ambiente 3D.
        \end{itemize}
    }
    { % DESCRIZIONE
        \begin{itemize}
            \item Visualizzazione del dataset caricato in forma tabellare.
        \end{itemize}
    }
    { % PRECONDIZIONI
        \begin{itemize}
            \item L'utente è all'interno dell'ambiente 3D;
            \item Il caricamento del dataset è terminato.
        \end{itemize}
    }
    { % POSTCONDIZIONI
        \begin{itemize}
            \item La tabella contenente i dati del dataset è visibile con i rispettivi valori.
        \end{itemize}
    }
    { % SCENARIO PRIMARIO
        \begin{itemize}
            \item Nessuna azione richiesta.
        \end{itemize}
    }


\SubUseCase{Visualizza intestazioni}
\UCdsc
    { % ATTORE
        \begin{itemize}
            \item Utente in ambiente 3D.
        \end{itemize}
    }
    { % DESCRIZIONE
        \begin{itemize}
            \item Visualizzazione delle intestazioni della tabella contenente i tutti i valori del dataset selezionato.
        \end{itemize}
    }
    { % PRECONDIZIONI
        \begin{itemize}
            \item La tabella è visibile;
        \end{itemize}
    }
    { % POSTCONDIZIONI
        \begin{itemize}
            \item Le intestazioni della tabella sono visibili.
        \end{itemize}
    }
    { % SCENARIO PRIMARIO
        \begin{itemize}
            \item Nessuna azione richiesta.
        \end{itemize}
    }

\SubUseCase{Visualizza dati}
\UCdsc
    { % ATTORE
        \begin{itemize}
            \item Utente in ambiente 3D.
        \end{itemize}
    }
    { % DESCRIZIONE
        \begin{itemize}
            \item Visualizzazione di tutti i valori del dataset selezionato attraverso le celle della tabella.
        \end{itemize}
    }
    { % PRECONDIZIONI
        \begin{itemize}
            \item La tabella è visibile.
        \end{itemize}
    }
    { % POSTCONDIZIONI
        \begin{itemize}
            \item I valori del dataset selezionato vengono visualizzati nella tabella.
        \end{itemize}
    }
    { % SCENARIO PRIMARIO
        \begin{itemize}
            \item Nessuna azione richiesta.
        \end{itemize}
    }

\UseCase{Visualizza dataset tramite grafico 3D}
\begin{figure}[h!]
    \centering
    \includegraphics[scale=0.45]{template/images/UC6.png}
    \caption{\nameref*{UC:\arabic{UC}}}
\end{figure}
\UCdsc
    { % ATTORE
        \begin{itemize}
            \item Utente in ambiente 3D.
        \end{itemize}
    }
    { % DESCRIZIONE
        \begin{itemize}
            \item Visualizzazione del dataset caricato sotto forma di grafico 3D a barre verticali.
        \end{itemize}
    }
    { % PRECONDIZIONI
        \begin{itemize}
            \item L'utente è all'interno dell'ambiente 3D;
            \item Il caricamento del dataset `e terminato.
        \end{itemize}
    }
    { % POSTCONDIZIONI
        \begin{itemize}
            \item Il grafico 3D è visibile.
        \end{itemize}
    }
    { % SCENARIO PRIMARIO
        \begin{itemize}
            \item Nessuna azione richiesta.
        \end{itemize}
    }




\SubUseCase{Visualizza assi}
\UCdsc
    { % ATTORE
        \begin{itemize}
            \item Utente in ambiente 3D.
        \end{itemize}
    }
    { % DESCRIZIONE
        \begin{itemize}
            \item Visualizzazione degli asse X, Y e Z del grafico 3D.
        \end{itemize}
    }
    { % PRECONDIZIONI
        \begin{itemize}
            \item Il grafico 3D è visibile.
        \end{itemize}
    }
    { % POSTCONDIZIONI
        \begin{itemize}
            \item Gli assi del grafico 3D sono visibili.
        \end{itemize}
    }
    { % SCENARIO PRIMARIO
        \begin{itemize}
            \item Nessuna azione richiesta.
        \end{itemize}
    }

\SubSubUseCase{Visualizza asse X}
\UCdsc
    { % ATTORE
        \begin{itemize}
            \item Utente in ambiente 3D.
        \end{itemize}
    }
    { % DESCRIZIONE
        \begin{itemize}
            \item Visualizzazione degli asse X del grafico 3D con le sue eventuali etichette.
        \end{itemize}
    }
    { % PRECONDIZIONI
        \begin{itemize}
            \item Il grafico 3D è visibile.
        \end{itemize}
    }
    { % POSTCONDIZIONI
        \begin{itemize}
            \item L'asse X del grafico 3D è visibile;
            \item Le etichette dell'asse X del grafico 3D sono visibili.
        \end{itemize}
    }
    { % SCENARIO PRIMARIO
        \begin{itemize}
            \item Nessuna azione richiesta.
        \end{itemize}
    }

\SubSubUseCase{Visualizza asse Y}
\UCdsc
    { % ATTORE
        \begin{itemize}
            \item Utente in ambiente 3D.
        \end{itemize}
    }
    { % DESCRIZIONE
        \begin{itemize}
            \item Visualizzazione degli asse Y del grafico 3D con le sue eventuali etichette.
        \end{itemize}
    }
    { % PRECONDIZIONI
        \begin{itemize}
            \item Il grafico 3D è visibile.
        \end{itemize}
    }
    { % POSTCONDIZIONI
        \begin{itemize}
            \item L'asse Y del grafico 3D è visibile;
            \item Le etichette dell'asse Y del grafico 3D sono visibili.
        \end{itemize}
    }
    { % SCENARIO PRIMARIO
        \begin{itemize}
            \item Nessuna azione richiesta.
        \end{itemize}
    }

\SubSubUseCase{Visualizza asse Z}
\UCdsc
    { % ATTORE
        \begin{itemize}
            \item Utente in ambiente 3D.
        \end{itemize}
    }
    { % DESCRIZIONE
        \begin{itemize}
            \item Visualizzazione degli asse Z del grafico 3D con le sue eventuali etichette.
        \end{itemize}
    }
    { % PRECONDIZIONI
        \begin{itemize}
            \item Il grafico 3D è visibile.
        \end{itemize}
    }
    { % POSTCONDIZIONI
        \begin{itemize}
            \item L'asse Z del grafico 3D è visibile;
            \item Le etichette dell'asse Z del grafico 3D sono visibili.
        \end{itemize}
    }
    { % SCENARIO PRIMARIO
        \begin{itemize}
            \item Nessuna azione richiesta.
        \end{itemize}
    }


\UseCase{Spostamento orizzontale della telecamera}
    \begin{figure}[h!]
        \centering
        \includegraphics[scale=0.65]{template/images/UC7.png}
        \caption{\nameref*{UC:\arabic{UC}}}
    \end{figure}
    \UCdsc
        { % ATTORE
            \begin{itemize}
                \item Utente in ambiente 3D.
            \end{itemize}
        }
        { % DESCRIZIONE
            \begin{itemize}
                \item Spostare la telecamera orizzontalmente significa far scorrere la vista a destra o a sinistra, mantenendo la stessa inclinazione.
            \end{itemize}
        }
        { % PRECONDIZIONI
            \begin{itemize}
                \item Le azioni di movimento devono essere valide. Un'azione di movimento è valida se non va oltre il margine destro o sinistro di un'area predefinita dal sistema;
                \item La telecamera si trova in una posizione arbitraria nello spazio.
            \end{itemize}
        }
        { % POSTCONDIZIONI
            \begin{itemize}
                \item  La telecamera si trova in una posizione orizzontale scelta dall'utente, diversa da quella iniziale.
            \end{itemize}
        }
        { % SCENARIO PRIMARIO
            \begin{itemize}
                \item L'utente interagisce con il sistema per compiere un'azione di movimento di spostamento orizzontale.
            \end{itemize}
        }

\UseCase{Spostamento verticale della telecamera}
    \begin{figure}[h!]
        \centering
        \includegraphics[scale=0.65]{template/images/UC8.png}
        \caption{\nameref*{UC:\arabic{UC}}}
    \end{figure}
    \UCdsc
        { % ATTORE
            \begin{itemize}
                \item Utente in ambiente 3D.
            \end{itemize}
        }
        { % DESCRIZIONE
            \begin{itemize}
                \item Spostare la telecamera verticale significa far scorrere la vista in alto o in basso, mantenendo la stessa inclinazione.
            \end{itemize}
        }
        { % PRECONDIZIONI
            \begin{itemize}
                \item Le azioni di movimento devono essere valide. Un'azione di movimento è valida se non va oltre il margine superiore o inferiore di un'area predefinita dal sistema;
                \item La telecamera si trova in una posizione arbitraria nello spazio.
            \end{itemize}
        }
        { % POSTCONDIZIONI
            \begin{itemize}
                \item La telecamera si trova in una posizione verticale scelta dall'utente, diversa da quella iniziale.
            \end{itemize}
        }
        { % SCENARIO PRIMARIO
            \begin{itemize}
                \item L'utente interagisce con il sistema per compiere un'azione di movimento di spostamento verticale.
            \end{itemize}
        }

\UseCase{Ruota grafico attorno asse X}
\begin{figure}[h!]
    \centering
    \includegraphics[scale=0.65]{template/images/UC9.png}
    \caption{\nameref*{UC:\arabic{UC}}}
\end{figure}
\UCdsc
{ % ATTORE
    \begin{itemize}
        \item Utente in ambiente 3D.
    \end{itemize}
}
{ % DESCRIZIONE
    \begin{itemize}
        \item L'attore compie un'azione di rotazione del grafico attorno l'asse X.
    \end{itemize}
}
{ % PRECONDIZIONI
    \begin{itemize}
        \item Il grafico è visibile;
        \item Il grafico ha una rotazione arbitraria.
    \end{itemize}
}
{ % POSTCONDIZIONI
    \begin{itemize}
        \item Il grafico ha una rotazione scelta dall'utente, diversa rispetto a quella iniziale.
    \end{itemize}
}
{ % SCENARIO PRIMARIO
    \begin{itemize}
        \item L'utente interagisce con il sistema per compiere di rotazione del grafico attorno all'asse X.
    \end{itemize}
}

\UseCase{Ruota grafico attorno asse Y}
\begin{figure}[h!]
    \centering
    \includegraphics[scale=0.65]{template/images/UC10.png}
    \caption{\nameref*{UC:\arabic{UC}}}
\end{figure}
\UCdsc
{ % ATTORE
    \begin{itemize}
        \item Utente in ambiente 3D.
    \end{itemize}
}
{ % DESCRIZIONE
    \begin{itemize}
        \item L'attore compie un'azione di rotazione del grafico attorno l'asse Y.
    \end{itemize}
}
{ % PRECONDIZIONI
    \begin{itemize}
        \item Il grafico è visibile;
        \item Il grafico ha una rotazione arbitraria.
    \end{itemize}
}
{ % POSTCONDIZIONI
    \begin{itemize}
        \item Il grafico ha una rotazione scelta dall'utente, diversa rispetto a quella iniziale.
    \end{itemize}
}
{ % SCENARIO PRIMARIO
    \begin{itemize}
        \item L'utente interagisce con il sistema per compiere di rotazione del grafico attorno all'asse Y.
    \end{itemize}
}

\UseCase{Ruota grafico attorno asse Z}
\begin{figure}[h!]
    \centering
    \includegraphics[scale=0.65]{template/images/UC11.png}
    \caption{\nameref*{UC:\arabic{UC}}}
\end{figure}
\UCdsc
{ % ATTORE
    \begin{itemize}
        \item Utente in ambiente 3D.
    \end{itemize}
}
{ % DESCRIZIONE
    \begin{itemize}
        \item L'attore compie un'azione di rotazione del grafico attorno l'asse Z.
    \end{itemize}
}
{ % PRECONDIZIONI
    \begin{itemize}
        \item Il grafico è visibile;
        \item Il grafico ha una rotazione arbitraria.
    \end{itemize}
}
{ % POSTCONDIZIONI
    \begin{itemize}
        \item Il grafico ha una rotazione scelta dall'utente, diversa rispetto a quella iniziale.
    \end{itemize}
}
{ % SCENARIO PRIMARIO
    \begin{itemize}
        \item L'utente interagisce con il sistema per compiere di rotazione del grafico attorno all'asse Z.
    \end{itemize}
}
    

