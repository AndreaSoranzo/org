\section{Introduzione}
    \subsection{Scopo del documento}
    Questo documento ha lo scopo di definire le norme e le linee guida operative per il team \textit{Six Bit Busters} nello sviluppo del progetto \textit{3Dataviz}.\\ 
    In particolare, esso descrive i processi di lavoro, le modalità di collaborazione, gli standard di codifica e le pratiche di gestione della qualità che il team seguirà per garantire coerenza, efficienza e qualità durante il ciclo di vita del prodotto. 
    Lo scopo è quello di fornire una struttura comrune e procedure chiare, per facilitare il lavoro di squadra, garantendo che tutti i membri operino in linea con gli obiettivi e le specifiche concordate.
ddfflsdsddddddddgfg
    \subsection{Scopo del prodotto}
    \textit{3Dataviz} è un prodotto ideato dall'azienda \textit{Sanmarco Informatica S.p.A.} per semplificare e rendere più accessibile la visualizzazione dei dati.\\
    Il progetto si basa sul concetto di data visualization, che consiste nel trasformare i dati in grafici e rappresentazioni visive, sfruttando la capacità del cervello umano di elaborare rapidamente le immagini. 
    Questo approccio facilita il processo decisionale e migliora la comprensione delle informazioni.\\
    L’obiettivo principale è lo sviluppo di un’interfaccia web che trasforma dati provenienti da diverse fonti (come database e REST API) in grafici 3D interattivi e navigabili. 
    Inoltre, i dati potranno essere consultati anche in formato tabellare, offrendo una visione alternativa ma altrettanto utile.  

    \subsection{Glossario}
    Per chiarire i termini tecnici o ambigui si utilizza un glossario disponibile nel file \textit{Glossario}.\\
    Tutti i termini che richiedono spiegazioni sono indicati con il pedice “g”. \\
    Questa convenzione consente un rapido collegamento tra il testo principale e la relativa spiegazione dettagliata nel glossario, garantendo coerenza e chiarezza nella comunicazione.

    \subsection{Riferimenti}
        \subsubsection{Riferimenti normativi}
        \begin{itemize}
            \item Capitolato d'appalto C5 \textit{Sanmarco Informatica S.p.A.} - 3Dataviz : \\ \url{https://www.math.unipd.it/~tullio/IS-1/2024/Progetto/C5.pdf}
            \item Materiale didattico - Corso Ingegneria del software 2024/2025 - Regolamento del Progetto Didattico: \\ \url{https://www.math.unipd.it/~tullio/IS-1/2024/Dispense/PD1.pdf}
        \end{itemize}
        \subsubsection{Riferimenti informativi}
        \begin{itemize}
            \item Materiale didattico - Corso Ingegneria del software 2024/2025 - Processi di ciclo di vita: \\ \url{https://www.math.unipd.it/~tullio/IS-1/2024/Dispense/T02.pdf}
            \item Materiale didattico - Corso Ingegneria del software 2024/2025 - Il ciclo di vita del SW: \\ \url{https://www.math.unipd.it/~tullio/IS-1/2024/Dispense/T03.pdf}
            \item Materiale didattico - Corso Ingegneria del software 2024/2025 - Gestione di progetto: \\ \url{https://www.math.unipd.it/~tullio/IS-1/2024/Dispense/T04.pdf}
            \item Materiale didattico - Corso Ingegneria del software 2024/2025 - Verifica e validazione: analisi dinamica: \\ \url{https://www.math.unipd.it/~tullio/IS-1/2024/Dispense/T11.pdf}
            \item Riferimento per alcune metriche di processo: \\ \url{https://it.wikipedia.org/wiki/Metriche_di_progetto}
            \item Indice Gulpease: \\ \url{https://it.wikipedia.org/wiki/Indice_Gulpease}
            \item Requirements Stability Index (RSI): \\ \url{https://shiyamtj.wordpress.com/2018/09/26/requirement-stability-index/}
        
        \end{itemize}
